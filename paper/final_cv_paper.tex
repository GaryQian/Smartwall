%%%%%%%%%%%%%%%%%%%%%%%%%%%%%%%%%%%%%%%%%
% Journal Article
% LaTeX Template
% Version 1.4 (15/5/16)
%
% This template has been downloaded from:
% http://www.LaTeXTemplates.com
%
% Original author:
% Frits Wenneker (http://www.howtotex.com) with extensive modifications by
% Vel (vel@LaTeXTemplates.com)
%
% License:
% CC BY-NC-SA 3.0 (http://creativecommons.org/licenses/by-nc-sa/3.0/)
%
%%%%%%%%%%%%%%%%%%%%%%%%%%%%%%%%%%%%%%%%%

%----------------------------------------------------------------------------------------
%	PACKAGES AND OTHER DOCUMENT CONFIGURATIONS
%----------------------------------------------------------------------------------------

\documentclass[twoside,twocolumn]{article}

\usepackage{blindtext} % Package to generate dummy text throughout this template 

\usepackage[sc]{mathpazo} % Use the Palatino font
\usepackage[T1]{fontenc} % Use 8-bit encoding that has 256 glyphs
\linespread{1.05} % Line spacing - Palatino needs more space between lines
\usepackage{microtype} % Slightly tweak font spacing for aesthetics

\usepackage[english]{babel} % Language hyphenation and typographical rules

\usepackage[hmarginratio=1:1,top=32mm,columnsep=20pt]{geometry} % Document margins
\usepackage[hang, small,labelfont=bf,up,textfont=it,up]{caption} % Custom captions under/above floats in tables or figures
\usepackage{booktabs} % Horizontal rules in tables

\usepackage{lettrine} % The lettrine is the first enlarged letter at the beginning of the text

\usepackage{enumitem} % Customized lists
\setlist[itemize]{noitemsep} % Make itemize lists more compact

\usepackage{abstract} % Allows abstract customization
\renewcommand{\abstractnamefont}{\normalfont\bfseries} % Set the "Abstract" text to bold
\renewcommand{\abstracttextfont}{\normalfont\small\itshape} % Set the abstract itself to small italic text

\usepackage{titlesec} % Allows customization of titles
\renewcommand\thesection{\Roman{section}} % Roman numerals for the sections
\renewcommand\thesubsection{\roman{subsection}} % roman numerals for subsections
\titleformat{\section}[block]{\large\scshape\centering}{\thesection.}{1em}{} % Change the look of the section titles
\titleformat{\subsection}[block]{\large}{\thesubsection.}{1em}{} % Change the look of the section titles

\usepackage{fancyhdr} % Headers and footers
\pagestyle{fancy} % All pages have headers and footers
\fancyhead{} % Blank out the default header
\fancyfoot{} % Blank out the default footer
\fancyhead[C]{EN 600.461: Computer Vision $\bullet$ December 2016 } % Custom header text
\fancyfoot[RO,LE]{\thepage} % Custom footer text

\usepackage{titling} % Customizing the title section

\usepackage{hyperref} % For hyperlinks in the PDF

%----------------------------------------------------------------------------------------
%	TITLE SECTION
%----------------------------------------------------------------------------------------

\setlength{\droptitle}{-4\baselineskip} % Move the title up

\pretitle{\begin{center}\Huge\bfseries} % Article title formatting
\posttitle{\end{center}} % Article title closing formatting
\title{SmartWall} % Article title
\author{%
\textsc{Gary Qian, Manyu Sharma, Tony Jiang, Sarah Sukardi}\\[1ex] % Your name
\normalsize Johns Hopkins University, Department of Computer Science \\ % Your institution
%\normalsize \href{mailto:john@smith.com}{john@smith.com} % Your email address
%\and % Uncomment if 2 authors are required, duplicate these 4 lines if more
%\textsc{Jane Smith}\thanks{Corresponding author} \\[1ex] % Second author's name
%\normalsize University of Utah \\ % Second author's institution
%\normalsize \href{mailto:jane@smith.com}{jane@smith.com} % Second author's email address
}


\date{\today} % Leave empty to omit a date
\renewcommand{\maketitlehookd}{%
\begin{abstract}
\noindent 
This paper presents Smartwall, a program that uses computer vision to turn any surface into an interactive board using a standard camera (such as a webcam) and a projector. The program uses a custom calibration matrix and perspective projection to enable object tracking, as well as employs deep learning for hand recognition and gestural board manipulation. Smartwall is an extremely accurate, cost-effective, and simple way to facilitate interactive teaching, brainstorming, and entertainment, at a fraction of the cost of other existing devices.
\end{abstract}
}

%----------------------------------------------------------------------------------------

\begin{document}

% Print the title
\maketitle

%----------------------------------------------------------------------------------------
%	ARTICLE CONTENTS
%----------------------------------------------------------------------------------------

\section{Introduction}

\lettrine[nindent=0em,lines=3]{H} umans have drawn on surfaces for millenia. From primitive pre-historic cave paintings to 17th century frescoes to the chalkboards and whiteboards commonly used in schools and universities today, the usage of surfaces as conduits for brainstorming, depicting information, and even art, have made them long essential to processes of creativity and conveyance.\\ \\Current, modern approaches to drawing on surfaces suffer from either requiring physical media (whiteboards, chalkboards, pens) or expensive equipment (modern-day smartboards). This paper presents a cost-effective method to turn any wall into a drawable surface using only two pieces of equipment: a camera and a projector, where the projector can be substituted with any medium capable of displaying digital content (ie. televisions, monitors, etc.). \\ \\ Our method is easily adaptable to spaces with unique constraints and requires equipment that most modern rooms already come equipped with, combining both traditional as well as state-of-the art computer vision techniques to allow for sophisticated and accurate gestural recognition.
%------------------------------------------------
\section{Methods}

\subsection{Overview}
We separate our discussion of methods into several subsections:
\begin{itemize}
\item Calibration
\item Training
\item Detection
\item Recognition
\item Output
\end{itemize}
\subsection{Calibration}
For camera calibration, the camera used to track hand movement is set to point towards the wall. The projector also projects the content eventually to be controlled with a human hand in the same direction as the camera. A custom pattern of green dots is displayed onto the wall for the camera to record; this is the setup required for the camera calibration process to begin.\\ \\
The camera then begins the process of calibration. Frames are captured in real-time from the camera, and then applied with a box blur with an 11-pixel radius. Each frame is then converted to HSV color-space to detect hues on the screen. The range of color to be detected is then defined in HSV and thresholded to retrieve only the desired values. Finally, the thresholded mask is eroded to remove noise.\\ \\
From each processed frame, contours are drawn and the center of each contour is computed, with nearby pixels within a 30-pixel radius clustered to form a single point. The amount of detected points is then gathered from the image. \\ \\
When the amount of detected points is equal to the amount of points on the custom pattern designed for optimum calibration, the points are sorted and then a homography is found to output a transformation matrix. If the pattern is not adequately detected, the system continues again for up to 13350 iterations.\\ \\
We have found that our system accurately detects our custom pattern projected onto a flat white background in less than 10 iterations, or frames, even with the camera positioned at various angles.\footnote{Example footnote}.

%------------------------------------------------

\section{Results}

\begin{table}
\caption{Example table}
\centering
\begin{tabular}{llr}
\toprule
\multicolumn{2}{c}{Name} \\
\cmidrule(r){1-2}
First name & Last Name & Grade \\
\midrule
John & Doe & $7.5$ \\
Richard & Miles & $2$ \\
\bottomrule
\end{tabular}
\end{table}

\blindtext % Dummy text

\begin{equation}
\label{eq:emc}
e = mc^2
\end{equation}

\blindtext % Dummy text

%------------------------------------------------

\section{Discussion}

\subsection{Subsection One}

A statement requiring citation \cite{Figueredo:2009dg}.
\blindtext % Dummy text

\subsection{Subsection Two}

\blindtext % Dummy text

%----------------------------------------------------------------------------------------
%	REFERENCE LIST
%----------------------------------------------------------------------------------------

\begin{thebibliography}{99} % Bibliography - this is intentionally simple in this template

%item 1
\bibitem[Figueredo and Wolf, 2009]{Figueredo:2009dg}
Figueredo, A.~J. and Wolf, P. S.~A. (2009).
\newblock Assortative pairing and life history strategy - a cross-cultural
  study.
\newblock {\em Human Nature}, 20:317--330.
 
 %item 2
 \bibitem[Figueredo and Wolf, 2009]{Figueredo:2009dg}
 Figueredo, A.~J. and Wolf, P. S.~A. (2009).
 \newblock Assortative pairing and life history strategy - a cross-cultural
 study.
 \newblock {\em Human Nature}, 20:317--330.
\end{thebibliography}

%----------------------------------------------------------------------------------------

\end{document}
